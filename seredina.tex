
\documentclass[a4paper,12pt]{article}
%%% Работа с русским языком
\usepackage{cmap}					% поиск в PDF
\usepackage{mathtext} 				% русские буквы в фомулах
\usepackage[T2A]{fontenc}			% кодировка
\usepackage[utf8]{inputenc}			% кодировка исходного текста
\usepackage[english,russian]{babel}	% локализация и переносы

%%% Дополнительная работа с математикой
\usepackage{amsfonts,amssymb,amsthm,mathtools} % AMS
\usepackage{amsmath}
\usepackage{amsfonts,amssymb,amsthm,mathtools} % AMS
\usepackage{icomma} % "Умная" запятая: $0,2$ --- число, $0, 2$ --- перечисление
\everymath{\displaystyle}
%% Номера формул
%\mathtoolsset{showonlyrefs=true} % Показывать номера только у тех формул, на которые есть \eqref{} в тексте.

%% Шрифты
\usepackage{euscript}	 % Шрифт Евклид
\usepackage{mathrsfs} % Красивый матшрифт

%% Свои команды
\DeclareMathOperator{\sgn}{\mathop{sgn}}

%% Перенос знаков в формулах (по Львовскому)
\newcommand*{\hm}[1]{#1\nobreak\discretionary{}
	{\hbox{$\mathsurround=0pt #1$}}{}}

%%% Работа с картинками
\usepackage{graphicx}  % Для вставки рисунков
\graphicspath{{Изображения/}{image}}  % папки с картинками
\setlength\fboxsep{3pt} % Отступ рамки \fbox{} от рисунка
\setlength\fboxrule{1pt} % Толщина линий рамки \fbox{}
\usepackage{wrapfig} % Обтекание рисунков и таблиц текстом

%%% Работа с таблицами
\usepackage{array,tabularx,tabulary,booktabs} % Дополнительная работа с таблицами
\usepackage{longtable}  % Длинные таблицы
\usepackage{multirow} % Слияние строк в таблице
\usepackage[usenames,dvipsnames]{color}

\usepackage{alltt}

\usepackage{color} %% это для отображения цвета в коде
\usepackage[dvipsnames]{xcolor}
\usepackage{listings} %% собственно, это и есть пакет listings
\begin{document}
	
	\title{Важные функции}
	\section{Работа с массивами}
	\subsection{Многочлены}
	Элементы массива - коэффициенты, стоящие в порядке возрастания степеней.
	\begin{alltt}
		\textcolor{Orange}{Array}.\textcolor{Blue}{prototype}.\textcolor{Purple}{mn_proizv}=\textcolor{Red}{function()}
	\end{alltt}
	Находит производную от многочлена. 
	\begin{alltt}
		\textcolor{Orange}{Array}.\textcolor{Blue}{prototype}.\textcolor{Purple}{mn_vychisl}=\textcolor{Red}{function()}
	\end{alltt}
	Находит корни многочлена.
	\begin{alltt}
		\textcolor{Orange}{Array}.\textcolor{Blue}{prototype}.\textcolor{Purple}{mn_txt}=\textcolor{Red}{function()}
	\end{alltt}
	TeX-представление многочлена.%%%???
	\begin{alltt}
		\textcolor{Orange}{Array}.\textcolor{Blue}{prototype}.\textcolor{Purple}{mn_pervoobr}=\textcolor{Red}{function()}
	\end{alltt}
	Находит первообразную от многочлена.%%TODO в чёр разница то??
	\begin{alltt}
		\textcolor{Orange}{Array}.\textcolor{Blue}{prototype}.\textcolor{Purple}{mn_txtmas}=\textcolor{Red}{function}()
	\end{alltt}
	TeX-представление многочлена.
	\begin{alltt}
		\textcolor{Orange}{Array}.\textcolor{Blue}{prototype}.\textcolor{Purple}{mt_pryam}=\textcolor{Red}{function}()
	\end{alltt}
	Возвращает коэффициенты a и b прямой y=ax+b, проходящей через две первые точки.
	\begin{alltt}
		\textcolor{Orange}{Array}.\textcolor{Blue}{prototype}.\textcolor{Purple}{shuffle}=\textcolor{Red}{function}()
	\end{alltt}	
	Перемешивает массив случайным образом. Если b, то ещё и рекурсивно на один уровень.
	\begin{alltt}
		\textcolor{Orange}{Array}.\textcolor{Blue}{prototype}.\textcolor{Purple}{iz}=\textcolor{Red}{function}(p1)
	\end{alltt}
	Если p1 опущено, возвращает случайный элемент массива, иначе последовательность p1 неповторяющихся элементов массива.
	\section{Работа с canvas}
	\begin{alltt}
		\textcolor{Orange}{CanvasRenderingContext2D}.\textcolor{Blue}{prototype}.\textcolor{Purple}{drawLine}=\textcolor{Red}{function}(x1,y1,x2,y2)
	\end{alltt}
	Рисует линию из точки (x1,y1) в (x2,y2).
	\begin{alltt}
		\textcolor{Orange}{CanvasRenderingContext2D}.\textcolor{Blue}{prototype}.\textcolor{Purple}{fillKrug}=\textcolor{Red}{function}(x,y,r)
	\end{alltt}
	Рисует круг с центром в (x,y) и радиусом r. 
		\begin{alltt}
		\textcolor{Orange}{CanvasRenderingContext2D}.\textcolor{Blue}{prototype}.\textcolor{Purple}{drawArrow}=\textcolor{Red}{function}(x1, y1, x2, y2, arrowType)
	\end{alltt}%%TODO Узнать зачем arrowType
	Рисует стрелку из точки (x1,y1) в (x2,y2)
	\begin{alltt}
		\textcolor{Orange}{CanvasRenderingContext2D}.\textcolor{Blue}{prototype}.\textcolor{Purple}{drawCoordPlane }=\textcolor{Red}{function}(w, h, kl, text, mash)
	\end{alltt}
	Рисует координатную плоскость. w и h  \-- её размеры, kl \-- объект с полями hor и ver(высота и ширина клетки), text \-- объект с полями x1 и y1(единичные отрезки типа string), sh1 и sh2 (шрифты для x1, y1, по умолчанию 12px) и mash - масштаб изображения (по умолчанию равно 1).
	\begin{alltt}
		\textcolor{Orange}{CanvasRenderingContext2D}.\textcolor{Blue}{prototype}.\textcolor{Purple}{setkaVer2  }=\textcolor{Red}{function}(h, w, hor, ver, mash)
	\end{alltt}
	Рисует прямоугольную сетку. w и h  \-- её размеры, hor и ver \-- высота и ширина клетки, mash - масштаб (по умолчанию равно 1).
	\section{Работа с матрицами}

	\begin{alltt} 	\textcolor{Red}{function} \textcolor{Purple}{multiplyMatrix}(A,B)
	\end{alltt}
	Умножает матрицу A на B, возвращает результат в матрице C.
	\begin{alltt} 	\textcolor{Red}{function} \textcolor{Purple}{Determinant}(A)
	\end{alltt}
	Возвращает определитель матрицы A.
	\begin{alltt} 	\textcolor{Red}{function} \textcolor{Purple}{MatrixCofactor}(i,j,A)
	\end{alltt}
	Возвращает алгебраическое дополнение матрицы A.
	\begin{alltt} 	\textcolor{Red}{function} \textcolor{Purple}{AdjugateMatrix}(A)
	\end{alltt}
	Возвращает союзную(присоединенную) матрицу.
	\begin{alltt} 	\textcolor{Red}{function} \textcolor{Purple}{InverseMatrix}(B)
	\end{alltt}
	Возвращает обратную  к B матрицу.
		\begin{alltt} 	\textcolor{Red}{function} \textcolor{Purple}{generateMatrix}(stroki,stolbcy,min,max,p1)
	\end{alltt}
	Генерирует матрицу из случайных чисел.
	\section{Вспомогательные функции}
	\begin{alltt} 	\textcolor{Red}{function} \textcolor{Purple}{sluchch}(n,k,s)
	\end{alltt}
	Возвращает случайное число от n до k с шагом s(по умолчанию 1).
	\begin{alltt} 	\textcolor{Red}{function} \textcolor{Purple}{slKrome}(a,p1,p2,p3)
	\end{alltt}
	Возвращает случайное число, кроме a. Если a \-- массив, то не содержащееся в нём; Если число или строка, то не равное ему; Если функция, принимающая параметр - то не удовлетворяющее ей.
	\begin{alltt} 	\textcolor{Red}{function} \textcolor{Purple}{sluchDel}(a)
	\end{alltt}
	Случайный делитель числа a.
	\begin{alltt} 	\textcolor{Red}{function} \textcolor{Purple}{sluchiz}(a,n)
	\end{alltt}
	Возвращает массив из n случайных повторяющихся элементов массива a. 
\end{document}
