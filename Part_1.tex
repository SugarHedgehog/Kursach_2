
%%Программная реализация (на языке Javascript) алгоритмов
%% генерации фонда оценочных средств по математике
\section{Глава первая}
\subsection{Реализация алгоритмов на базе проекта Час ЕГЭ}
Как было уже сказано ранее, генерация заданий происходит при помощи шаблонов.
Они состоят из нескольких частей: непосредственно текстового задания, программной реализации на языке JavaScript,
графической на основе canvas\footnote{HTML элемента, использующегося для рисования графики средствами языков программирования и алгоритма решения задачи} в случае необходимости.

Алгоритм написания простого шаблона~\cite{chasi} на основе номера 109083 Открытого Банка Заданий ЕГЭ~\cite{fipi}:
\begin{enumerate}
    \item Выбираем задание из Открытого Банка Заданий ЕГЭ и копируем его текст.
    \item Добавляем ответ в поле answers (по умолчанию 0).
          \lstinputlisting{codes/listing_1.js}
    \item Инициализируем всех необходимые переменные для задачи (вес, проценты и так далее).
          Присваиваем им значения при помощи функции \hyperlink{sluchch}{sluchch()} или
          \hyperlink{slKrome}{slKrome()}. Для хранения ответа создаём отдельную переменную.
          \lstinputlisting{codes/listing_2.js}
    \item Заменяем все числа в тексте на переменные (при помощи ++).
    \item Обособляем слова, которые не влияют на условия задачи. Это могут быть имена, профессии, транспорт и тп.
    \item Создаём переменные, которые будет отвечать за выбранные в прошлом пункте слова, и заменяем слова на переменные в тексте задачи.
          Выбираем их значения из массивов при помощи \hyperlink{iz}{iz()}.
    \item Иногда в задании выбранные слова используются в разных падежах. Для этого в проекте существует лексический модуль. Используем на склоняемых словах функцию \hyperlink{sklonlxkand}{sklonlxkand()}. Теперь необходимо указать падеж слов в задании.
          Также при необходимости заглавной буквы в слове используем \hyperlink{toZagl}{toZagl()}.
    \item Если в тексте задачи присутствуют слова, зависимые от числительных, к ним применяется функция \hyperlink{chislitlx}{chislitlx()}.
          \lstinputlisting{codes/listing_3.js}
\end{enumerate}

Но не все задачи решаются при помощи линейного алгоритма. Тогда в коде шаблона
используется слишком много циклов с постусловиями(или предусловиями)
Для таких случаев в проекте существует окружение retryWhileUndefined(), которое
будет запускать программу до тех пор, пока случайные переменные не будут удовлетворять всем условиям.
Перезапуск осуществляется посредством постановки условий с последующим return.
Рассмотрим такой пример.

Алгоритм написания шаблона с изображением графика прямой:
\begin{enumerate}
    \item Для отображения рисунка необходимо вызвать функцию chas2.task.modifiers.

          addCanvasIllustration(), которой передаются высота и ширина изображения и имя функции отрисовки.
          \lstinputlisting{codes/listing_4.js}
    \item Пусть наша прямая проходит через две точки $(x_1,y_1)$ и $(x_2,y_2)$. Инициализируем их координаты случайными целыми числами. Теперь через них выразим коэффициенты уравнения прямой $y=kx+b$.
          Для дальнейшего вычисления точек прямой напишем вспомогательную функцию f(x).
          \lstinputlisting{codes/listing_5.js}
    \item Для того, чтобы уравнения нашей прямой можно было восстановить по рисунку необходимо и достаточно,
          чтобы были видны две её точки. Для поиска таких точек используем функцию \hyperlink{intPoints}{intPoints()}.
          \lstinputlisting{codes/listing_6.js}
    \item Приступим к изображению графика. Инициализируем переменную paint1 как функцию function(ct). Внутри неё зададим высоту и ширину графика в пикселях.
          Изображаем координатную плоскость функцией \hyperlink{drawCoordPlane}{drawCoordPlane()} и сам график функцией
          \hyperlink{graph9AdrawFunction}{graph9AdrawFunction()}. А также отметим две целые точки посредством \hyperlink{graph9AmarkCircles}{graph9AmarkCircles()}.
          \vspace{\baselineskip}

          \lstinputlisting{codes/listing_7.js}
\end{enumerate}
\subsection{Преимущества программной генерации заданий}
На примере двух предыдущих задач были явно показано превосходство шаблонов над заданиями из Открытого Банка Заданий, а именно:
\begin{enumerate}
    \item Большое количество разнообразных задач одного типа.
    \item Простота и скорость написания шаблонов.
    \item Невозможность нахождения учащимися ответов на задачи.
    \item Возможность юмористических формулировок заданий, что понижает моральное напряжение учащихся на проверочных работах.%%Лишний повод поржать
\end{enumerate}

