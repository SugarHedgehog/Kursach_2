\documentclass[a4paper,14pt]{article}
\usepackage{cmap}					% поиск в PDF
\usepackage{mathtext} 				% русские буквы в фомулах
\usepackage[T2A]{fontenc}			% кодировка
\usepackage[utf8]{inputenc}			% кодировка исходного текста
\usepackage[english,russian]{babel}	% локализация и переносы
\usepackage[left=25mm, top=20mm, right=10mm, bottom=20mm]{geometry}
\begin{document}
	\section{Часть первая}
	Проект "Час ЕГЭ" стал возможен ввиду стандартности заданий ЕГЭ. Так, используя один прототип обзада(Открытого банка заданий), автор пишет шаблон охватывающий огромное множество подобных, сохраняя их структуру и тип ответа. При этом возможна не только смена наполнения(имён, вид средств перемещения, цифр условия и т.п.), но и постановки вопроса. Который будет смежен с запрашиваемым в исходном. К примеру, прототип 26594 запрашивает: "Сколько деталей делает второй рабочий?". Шаблон может задать, как этот вопрос, так и: сумму, разность, произведение скоростей работы. Такие вопросы тренируют внимательность школьников на этапе подготовки к экзамену. 
 	\end{document}
