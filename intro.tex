
\section*{Введение}
\addcontentsline{toc}{section}{Введение}
Единый государственный экзамен (ЕГЭ)~\cite{ege} --- централизованно проводимый в Российской 
Федерации экзамен в средних учебных заведениях — школах, лицеях и гимназиях,
форма проведения ГИА(Государственный Итоговая Аттестация) по образовательным программам среднего общего образования.
Служит одновременно выпускным экзаменом из школы и вступительным экзаменом в вузы.

За два года подготовки к ЕГЭ школьники сталкиваются с дефицитом заданий для подготовки.
А учителя со списыванием ответов при решении задач экзамена учениками. 
Также в в конце 2021 года в список заданий ЕГЭ были добавлены новые задания под номером 9, 
количество которых для прорешивания очень мало. 
Проект «Час ЕГЭ» позволяет решить все эти проблемы.

«Час ЕГЭ» — компьютерный образовательный проект, разрабатываемый при математическом 
факультете ВГУ в рамках «OpenSource кластера» и предназначенный для помощи учащимся 
старших классов подготовиться к тестовой части единого государственного экзамена.
%%ссылочки на доклады
Задания в «Час ЕГЭ» генерируются случайным образом по специализированным алгоритмам, 
называемых шаблонами, каждый из которых
 охватывает множество вариантов соответствующей ему задачи. Для 
пользователей 
предназначены четыре оболочки (режима работы): «Случайное задание», «Тесты на печать», 
«Полный тест» и «Мини-интеграция».
«Час ЕГЭ» является полностью открытым (код находится под лицензией GNU GPL 3.0) 
и бесплатным.
В настоящее время в проекте полностью реализованы тесты по математике с кратким 
ответом (бывшая «часть В»).~\cite{fipi}
Планируется с течением времени включить в проект тесты по другим предметам школьной 
программы.

«Мини-интеграция» — это форма сотрудничества с образовательными интернет-ресурсами, 
при которой учебно-методический материал на странице ресурса дополняется виджетами 
тренажера с заданиями, соответствующими теме статьи, для возможности практического 
применения полученных знаний.
В настоящее время достигнуто сотрудничество с двумя образовательными ресурсами: ege-ok.ru 
и matematikalegko.ru.
