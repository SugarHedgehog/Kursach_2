\section{Глава вторая}\label{2sect}
\subsection{Функции, используемые в проекте}
За 9 лет работы над проектом «Час ЕГЭ» была разработан нестандартная библиотека для упрощения многих задач. Далее представлены наиболее используемые функции из неё.

\textbf{Работа с массивами}

\textbf{Многочлены}

Элементы одномерного массива Array --- коэффициенты, стоящие в порядке возрастания степеней.

\texttt{\textcolor{Orange}{Array}.\textcolor{Blue}{prototype}.\textcolor{Purple}{mn\_proizv}=\textcolor{Red}{function}()}

Находит производную от многочлена.

\texttt{\textcolor{Orange}{Array}.\textcolor{Blue}{prototype}.\textcolor{Purple}{mn\_vychisl}=\textcolor{Red}{function}()
}

Находит корни многочлена.

\texttt{
	\textcolor{Orange}{Array}.\textcolor{Blue}{prototype}.\textcolor{Purple}{mn\_txt}=\textcolor{Red}{function}()}

TeX-представление многочлена.%%%???

\texttt{
	\textcolor{Orange}{Array}.\textcolor{Blue}{prototype}.\textcolor{Purple}{mn\_pervoobr}=\textcolor{Red}{function}()
}

Находит первообразную от многочлена.%%TODO в чёр разница то??

\texttt{
	\textcolor{Orange}{Array}.\textcolor{Blue}{prototype}.\textcolor{Purple}{mn\_txtmas}=\textcolor{Red}{function}()
}

TeX-представление многочлена.

\texttt{
	\textcolor{Orange}{Array}.\textcolor{Blue}{prototype}.\textcolor{Purple}{mt\_pryam}=\textcolor{Red}{function}()
}

Возвращает коэффициенты $a$ и $b$ прямой $y=ax+b$, проходящей через две первые точки.

\textbf{Вспомогательные функции для массивов}

\texttt{
	\textcolor{Orange}{Array}.\textcolor{Blue}{prototype}.\textcolor{Purple}{shuffle}=\textcolor{Red}{function}(b)
}

Перемешивает массив случайным образом. Если b, то ещё и рекурсивно на один уровень.

\hypertarget{iz}{\texttt{
		\textcolor{Orange}{Array}.\textcolor{Blue}{prototype}.\textcolor{Purple}{iz}=\textcolor{Red}{function}(p1)
	}}

Если p1 опущено, возвращает случайный элемент массива, иначе массив p1 неповторяющихся элементов массива.

\textbf{Матрицы}

\texttt{
	\textcolor{Red}{function} \textcolor{Purple}{multiplyMatrix}(A,B)
}

Умножает матрицу A на B, возвращает результат в матрице C.

\texttt{
	\textcolor{Red}{function} \textcolor{Purple}{Determinant}(A)
}

Возвращает определитель матрицы A.

\texttt{
	\textcolor{Red}{function} \textcolor{Purple}{MatrixCofactor}(i,j,A)
}

Возвращает алгебраическое дополнение матрицы A.

\texttt{
	\textcolor{Red}{function} \textcolor{Purple}{AdjugateMatrix}(A)
}

Возвращает союзную(присоединенную) матрицу.

\texttt{
	\textcolor{Red}{function} \textcolor{Purple}{rang\_mat}(А)
}

Возвращает ранг матрицы A.

\texttt{
	\textcolor{Red}{function} \textcolor{Purple}{InverseMatrix}(B)
}

Возвращает обратную  к B матрицу.

\texttt{
	\textcolor{Red}{function} \textcolor{Purple}{generateMatrix}(stroki,stolbcy,min,max,p1)
}

Генерирует матрицу из случайных чисел.

\textbf{Работа с числами}

\hypertarget{chislitlx}{\texttt{
	\textcolor{Orange}{Number}.\textcolor{Blue}{prototype}.\textcolor{Purple}{chislitlx}=\textcolor{Red}{function}(p1,p2)
}}

Возвращает строку, состоящую из данного числа и подходящего падежа слова p1, при этом
полученное словосочетанию стоит в падеже p2 (если не указан - именительный).

\texttt{
	\textcolor{Orange}{Number}.\textcolor{Blue}{prototype}.\textcolor{Purple}{pow}=\textcolor{Red}{function}(n)
}

Возвращает число в степени n.

\texttt{
	\textcolor{Orange}{Number}.\textcolor{Blue}{prototype}.\textcolor{Purple}{sqrt}=\textcolor{Red}{function}(n)
}

Возвращает квадратный корень из числа.

\texttt{
	\textcolor{Orange}{Number}.\textcolor{Blue}{prototype}.\textcolor{Purple}{sqr}=\textcolor{Red}{function}()
}

Возвращает квадрат числа.

\texttt{
	\textcolor{Orange}{Number}.\textcolor{Blue}{prototype}.\textcolor{Purple}{abs}=\textcolor{Red}{function}()
}

Возвращает модуль числа.

\texttt{
	\textcolor{Orange}{Number}.\textcolor{Blue}{prototype}.\textcolor{Purple}{floor}=\textcolor{Red}{function}()
}

Возвращает число, округленное до целого в меньшую сторону.

\texttt{
	\textcolor{Orange}{Number}.\textcolor{Blue}{prototype}.\textcolor{Purple}{ceil}=\textcolor{Red}{function}()
}

Возвращает число, округленное до целого в большую сторону.

\texttt{
	\textcolor{Orange}{Number}.\textcolor{Blue}{prototype}.\textcolor{Purple}{pm}=\textcolor{Red}{function}()
}

Случайным образом возвращает число или ему противоположное.

\texttt{
	\textcolor{Orange}{Number}.\textcolor{Blue}{prototype}.\textcolor{Purple}{ts}=\textcolor{Red}{function}()
}
%%зачем? и пример
Приводит число к стандартному виду (с десятичной запятой и не более чем 10 знаками после неё).
Необходимо для отсечения ложной точности.

Пример:%%0.1+0.2
\begin{lstlisting}[frame=none]
	let number1=0.7+0.1;
	let number2=(0.7+0.1).ts()
\end{lstlisting}
\fbox{
	\parbox{10cm}{
		$number1=0.7999999999999999$

		$number2=0,8$
	}}

	\vspace{\baselineskip}
\texttt{
	\textcolor{Orange}{Number}.\textcolor{Blue}{prototype}.\textcolor{Purple}{texfracpi}=\textcolor{Red}{function}(p1)
}

Возвращает TeX-представление дроби, у которой в числителе данное число, умноженное на $\pi$, а в знаменателе p1.
Случай p1=1 учитывается.

\texttt{
	\textcolor{Orange}{Number}.\textcolor{Blue}{prototype}.\textcolor{Purple}{texsqrt}=\textcolor{Red}{function}(p1,p2)
}

TeX-представление корня из данного числа.
Если данное число - полный квадрат, то корень из числа.
Если p1, то из-под корня будут вынесены возможные множители.
Если p1, p2 и из-под корня выносится единица, то она будет опущена.

\texttt{
	\textcolor{Orange}{Number}.\textcolor{Blue}{prototype}.\textcolor{Purple}{isZ}=\textcolor{Red}{function}()
}

Возвращает true, если число n целое.

\texttt{
	\textcolor{Orange}{Number}.\textcolor{Blue}{prototype}.\textcolor{Purple}{isPolnKvadr}=\textcolor{Red}{function}()
}

Возвращает true, если число является полным квадратом.

\textbf{Работа со строками}

\hypertarget{toZagl}{\texttt{
		\textcolor{Orange}{Number}.\textcolor{Blue}{prototype}.\textcolor{Purple}{toZagl}=\textcolor{Red}{function}()
	}}

Возвращает исходную строку с первой заглавной буквой.

\texttt{
	\textcolor{Orange}{Number}.\textcolor{Blue}{prototype}.\textcolor{Purple}{esli}=\textcolor{Red}{function}()
}

Возвращает данную строку, если p1, и пустую в противном случае.

\texttt{
	\textcolor{Orange}{Number}.\textcolor{Blue}{prototype}.\textcolor{Purple}{plusminus}=\textcolor{Red}{function}()
}

Возвращает упрощенное выражение, вставляя между числами необходимые знаки и убирая нулевые.

\textbf{Работа с canvas}

\texttt{
	\textcolor{Orange}{CanvasRenderingContext2D}.\textcolor{Blue}{prototype}.\textcolor{Purple}{drawLine}=\textcolor{Red}{function}(x1,y1,x2,y2)
}

Рисует линию из точки (x1,y1) в (x2,y2).

\texttt{
	\textcolor{Orange}{CanvasRenderingContext2D}.\textcolor{Blue}{prototype}.\textcolor{Purple}{fillKrug}=\textcolor{Red}{function}(x,y,r)
}

Рисует круг с центром в (x,y) и радиусом r.

\texttt{
	\textcolor{Orange}{CanvasRenderingContext2D}.\textcolor{Blue}{prototype}.\textcolor{Purple}{drawArrow}=\textcolor{Red}{function}(x1, y1, x2, y2, arrowType)
}%%TODO Узнать зачем arrowType

Рисует стрелку из точки (x1,y1) в (x2,y2).

\hypertarget{drawCoordPlane}{\texttt{
		\textcolor{Orange}{CanvasRenderingContext2D}.\textcolor{Blue}{prototype}.\textcolor{Purple}{drawCoordPlane }=\textcolor{Red}{function}(w, h, kl, text, mash)
	}}

Рисует координатную плоскость. w и h  \--- её размеры, kl \--- объект с полями hor и ver (высота и ширина клетки), text \-- объект с полями x1 и y1(единичные отрезки типа string), sh1 и sh2 (шрифты для x1, y1, по умолчанию 12px) и mash - масштаб изображения (по умолчанию равно 1).

\texttt{
	\textcolor{Orange}{CanvasRenderingContext2D}.\textcolor{Blue}{prototype}.\textcolor{Purple}{setkaVer2}=
	\newline
	\textcolor{Red}{function}(h, w, hor, ver, mash)
}

Рисует прямоугольную сетку. w и h  \--- её размеры, hor и ver \--- высота и ширина клетки, mash - масштаб (по умолчанию равно 1).

\hypertarget{graph9AmarkCircles}{\texttt{
		\textcolor{Red}{function} \textcolor{Purple}{graph9AmarkCircles}(ct, XY, maxQuantity, radius)
	}}

Изображает не более maxQuantity точек в координатах из двумерно массива XY в виде кругов радиусом radius.
%%Переписать!

\hypertarget{graph9AdrawFunction}{\texttt{
		\textcolor{Red}{function} \textcolor{Purple}{graph9AdrawFunction}(ct, f, o)
	}}

Рисует график f(x). Границы отображения задаются объектом o с полями maxX, maxY, minX, minY.

\textbf{Вспомогательные функции}

\hypertarget{sluchch}{\texttt{
		\textcolor{Red}{function} \textcolor{Purple}{sluchch}(n,k,s)
	}}

Возвращает случайное число от n до k с шагом s (по умолчанию 1).
Эта функция используется настолько часто, что для неё была придумана сокращённая форма sl().

\hypertarget{slKrome}{\texttt{
		\textcolor{Red}{function} \textcolor{Purple}{slKrome}(a,p1,p2,p3)
	}}

Возвращает случайное число, кроме a. Если a \--- массив, то не содержащееся в нём; если число или строка, то не равное ему; Если функция, принимающая параметр - то не удовлетворяющее ей.

\texttt{
	\textcolor{Red}{function} \textcolor{Purple}{sluchDel}(a)
}

Возвращает случайный делитель числа a.

\texttt{
	\textcolor{Red}{function} \textcolor{Purple}{sluchiz}(a,n)
}

Возвращает массив из n случайных не повторяющихся элементов массива a.

\texttt{
	\textcolor{Red}{function} \textcolor{Purple}{slLetter}(b)
}

Возвращает случайную букву английского алфавита.%%зачем b?

\hypertarget{intPoints}{\texttt{
		\textcolor{Red}{function} \textcolor{Purple}{intPoints}(f,o)
	}}

Возвращает двумерный массив из всех целых точек графика f(x). Границы нахождения точек задаются объектом o с полями maxX, maxY, minX, minY.
\newpage
\subsection{Вклад автора в расширение каталога}
\lstinputlisting{codes/ex_109083.js}
Примеры генерируемых задач:

\vspace{\baselineskip}
\fbox{%
	\parbox{17cm}{%
		При сушке абрикоса получается курага. Сколько килограммов абрикоса
		потребуется для получения 66 килограммов кураги, если абрикос содержит
		90\% воды, а курага содержит 20\% воды?}%
}
\vspace{\baselineskip}

\fbox{%
	\parbox{17cm}{%
		При сушке дыни получается сухофрукт. Сколько килограммов дыни потребуется
		для получения 73 килограммов сухофрукта, если дыня содержит 88\% воды, а
		сухофрукт содержит 7\% воды?}%
}
\vspace{\baselineskip}

\fbox{%
	\parbox{17cm}{%
		При сушке винограда получается изюм. Сколько килограммов винограда
		потребуется для получения 70 килограммов изюма, если виноград содержит
		81\% воды, а изюм содержит 5\% воды?}%
}
\vspace{\baselineskip}
\newpage
\lstinputlisting{codes/ex_99621.js}
\newpage
Примеры генерируемых задач:

\vspace{\baselineskip}
\fbox{%
	\parbox{17cm}{%
		Максим и Арина выполняют одинаковую экзаменационную работу. Арина решает
		за час 7 заданий работы, а Максим 8. Они одновременно начали выполнять
		задания, и Арина закончила работу позже Максима на 60 минут. Сколько заданий
		содержит работа?}%
}

\vspace{\baselineskip}
\fbox{%
	\parbox{17cm}{%
		Матвей и Ирма выполняют одинаковую контрольную работу. Ирма решает за час
		9 примеров работы, а Матвей 10. Они одновременно начали решать примеры,
		и Ирма закончила работу позже Матвея на 20 минут. Сколько примеров содержит
		работа?}%
}

\vspace{\baselineskip}
\fbox{%
	\parbox{17cm}{%
		Коля и Женя выполняют одинаковый тест. Женя отвечает за час на 9 вопросов
		теста, а Коля 10. Они одновременно начали отвечать на вопросы, и Женя
		закончила тест позже Коли на 40 минут. Сколько вопросов содержит тест?}%
}
\vspace{\baselineskip}

\lstinputlisting{codes/ex_26775.js}
Примеры генерируемых задач:

\vspace{\baselineskip}
\fbox{%
	\parbox{17cm}{%
		Найдите $\ctg H$, если $\sin H=-\sqrt{\frac{9}{738}}$ и $H\in(37π;75π2)$.

		Найдём $\cos H\sqrt{1-(-\sqrt{\frac{9}{738}})^2=-\sqrt{\frac{729}{738}}}$(Угол H принадлежит
		III четверти - $\cos H<0$). Тогда $\ctg H=\sqrt{\frac{729}{9}}=9$}
}

\vspace{\baselineskip}
\fbox{%
	\parbox{17cm}{%
		Найдите $\tg M$, если $\cos M=\sqrt{\frac{4}{260}}$ и $M\in(\frac{101}{2}π;51π)$.

		Найдём $\sin M=\sqrt{1-(-\sqrt{\frac{9}{738}})^2=-\sqrt{\frac{729}{738}}}$(Угол  M принадлежит
		VI четверти - $sin M<0$). Тогда $\tg  M=\sqrt{\frac{256}{4}}=-8$}
}

\vspace{\baselineskip}
\fbox{%
	\parbox{17cm}{%
		Найдите $\tg Q$, если $\cos Q=-\sqrt{\frac{9}{153}}$ и $Q\in(\frac{171}{2}π;86π)$.

		Найдём $\sin = Q\sqrt{1-(-\sqrt{\frac{9}{153}})^2=\sqrt{\frac{144}{153}}}$(Угол  Q принадлежит
		VI четверти - $sin Q>0$). Тогда $\tg  Q=\sqrt{\frac{144}{9}}=-4$}
}
\vspace{\baselineskip}
\newpage
\lstinputlisting{codes/ex_26771.js}
Примеры генерируемых задач:

\vspace{\baselineskip}
\fbox{%
	\parbox{17cm}{%
		Найдите значение выражения $-72\tg201^{\circ}\cdot\tg249^{\circ}$.

		По формуле приведения $\tg249^{\circ}=\tg(450^{\circ}-249^{\circ})=\ctg201^{\circ}$.
		Тогда $-72\tg 201^{\circ}\cdot\ctg 201^{\circ}=-72$}}

\vspace{\baselineskip}
\fbox{%
\parbox{17cm}{%
Найдите значение выражения $-58\tg400^{\circ}\cdot \tg410^{\circ}$.

По формуле приведения $\tg410^{\circ}=\tg(810^{\circ}-410^{\circ})=\ctg400^{\circ}$.
Тогда $-58tg400^{\circ}\cdot\ctg400^{\circ}=-58$}}

\vspace{\baselineskip}
\fbox{%
	\parbox{17cm}{%
		Найдите значение выражения $-32\ctg195^{\circ}\cdot\ctg285^{\circ}.$

		По формуле приведения $\ctg285^{\circ}=\ctg(480^{\circ}-285^{\circ})=\tg195^{\circ}$.
		Тогда $-32\ctg195^{\circ}\cdot\tg195^{\circ}=-32$}}
\vspace{\baselineskip}
\newpage
\lstinputlisting{codes/ex_320169.js}
Примеры генерируемых задач:

\vspace{\baselineskip}
\fbox{%
	\parbox{17cm}{%
		Марина, Вика, Коля, Даниил, Арик, Надежда, Снежана и Наташа бросили жребий - кому начинать игру. Найдите вероятность того, что начинать игру должен будет мальчик.

		Жребий начать игру может выпасть каждому из восьми детей. Вероятность того, что это будет мальчик, равна 0.375.
	}}

\vspace{\baselineskip}
\fbox{%
	\parbox{17cm}{%
		Антон, Олег, Ира, Артём и Коля бросили жребий - кому начинать игру. Найдите вероятность того, что начинать игру должна будет не девочка.

		Жребий начать игру может выпасть каждому из пяти детей. Вероятность того, что это будет не девочка, равна 0.8.
	}}

\vspace{\baselineskip}
\fbox{%
	\parbox{17cm}{%
		Артём, Ирма, Данил, Антон и Максим бросили жребий - кому начинать игру. Найдите вероятность того, что начинать игру должна будет Ирма.

		Жребий начать игру может выпасть каждому из пяти детей. Вероятность того, что это будет Ирма, равна 0.2.
	}}

\vspace{\baselineskip}

\lstinputlisting{codes/ex_320207.js}
Примеры генерируемых задач:

\vspace{\baselineskip}
\fbox{
	\parbox{17cm}{
		Всем школьникам с подозрением на жмотство делают анализ крови. Если анализ
		выявляет жмотство, то результат анализа называется положительным. У больных
		жмотством школьников анализ даёт положительный результат в 66\%. Если школьники
		не болен жмотством, то анализ может дать ложный положительный результат с
		вероятностью 0,09. Известно, что 0,4 школьников, поступающих с подозрением на
		жмотство, действительно больны жмотством. Найдите вероятность того, что
		результат анализа у школьников, поступившего в клинику с подозрением на
		жмотство, будет положительным.

		Анализ школьников может быть положительным по двум причинам: 1) школьник
		болеет жмотством, его анализ верен; 2) пациент не болеет жмотством, его
		анализ ложен. По формуле условной вероятности: $66/100\cdot0,4=0,264$ и
		$0,09\cdot(1-0,4)=0,054$. События быть больным или быть здоровым образуют
		полную группу (они несовместны и одно из них непременно наступает), поэтому
		можно применить формулу полной вероятности. Тогда $0,264+0,054=0,318$. Ответ: $0,318$
	}}

\vspace{\baselineskip}
\fbox{
	\parbox{17cm}{
		Всем чиновникам с подозрением на диабет делают анализ крови. Если анализ
		выявляет диабет, то результат анализа называется положительным. У больных
		диабетом чиновников анализ даёт положительный результат с вероятностью 0,67.
		Если чиновники не болен диабетом, то анализ может дать ложный положительный
		результат в 9\%. Известно, что 73\% чиновников, поступающих с подозрением на
		диабет, действительно больны диабетом. Найдите процент того, что результат
		анализа у чиновников, поступившего в клинику с подозрением на диабет, будет отрицательным.

		Анализ чиновников может быть положительным по двум причинам: 1) чиновник
		болеет диабетом, его анализ верен; 2) пациент не болеет диабетом, его анализ
		ложен. По формуле условной вероятности: $0,67\cdot73/100=0,4891$ и
		$9/100\cdot(1-73/100)=0,0243$. События быть больным или быть здоровым
		образуют полную группу (они несовместны и одно из них непременно наступает),
		поэтому можно применить формулу полной вероятности. Тогда $0,4891+0,0243=0,5134$.
		Необходимо найти, что анализ будет отрицательным: $1-0,5134$. Так как нужно
		найти процент, умножим полученный ответ на 100: $0,5134\cdot100=48,66$ Ответ: $48,66$
	}}

\vspace{\baselineskip}
\fbox{
	\parbox{17cm}{
		Всем школьникам с подозрением на перфектолиз делают анализ крови. Если анализ
		выявляет перфектолиз, то результат анализа называется положительным.
		У больных перфектолизом школьников анализ даёт положительный результат в 91\%.
		Если школьники не болен перфектолизом, то анализ может дать ложный положительный
		результат с вероятностью $0,09$. Известно, что 0,47 школьников, поступающих с
		подозрением на перфектолиз, действительно больны перфектолизом. Найдите процент
		того, что результат анализа у школьников, поступившего в клинику с подозрением
		на перфектолиз, будет отрицательным.

		Анализ школьников может быть положительным по двум причинам: 1) школьник болеет
		перфектолизом, его анализ верен; 2) пациент не болеет перфектолизом, его анализ
		ложен. По формуле условной вероятности: $91/100\cdot0,47=0,4277$ и
		$0,09\cdot(1-0,47)=0,0477$. События быть больным или быть здоровым образуют
		полную группу (они несовместны и одно из них непременно наступает),
		поэтому можно применить формулу полной вероятности. Тогда 0,4277+0,0477=0,4754.
		Необходимо найти, что анализ будет отрицательным: $1-0,4754$. Так как нужно найти
		процент, умножим полученный ответ на 100: $0,4754\cdot100=52,46$ Ответ: $52,46$
	}}

\vspace{\baselineskip}

\lstinputlisting{codes/ex_50927103.js}
Примеры генерируемых задач:

\begin{figure}[h]
	\includegraphics[width=0.4\textwidth]{1.PNG}
\end{figure}
\fbox{
	\parbox{17cm}{
		На рисунке изображены графики функций $f(x)=a\sqrt{x-b}+c$ и
		$g(x)=kx+d$, которые пересекаются в точках A и B. Найдите абсциссу точки B.

		$f(x)=-2\sqrt{x+4}+2$

		$g(x)=-0,4x-1,2$

		$A(-3;0)$

		$B(12;-6)$
	}}

\vspace{\baselineskip}
\newpage
\begin{figure}[h]
	\includegraphics[width=0.4\textwidth]{2.PNG}
\end{figure}
\fbox{
	\parbox{17cm}{
		На рисунке изображены графики функций $f(x)=a\sqrt{x-b}+c$ и $g(x)=kx-d$, 
		которые пересекаются в точках A и B. Найдите абсциссу точки B.
		
		$f(x)=2\sqrt{x+5}-5$

		$g(x)=0,5x-2,5$
		
		$A(-5;-5)$

		$B(11;3)$
		}}

\vspace{\baselineskip}
\begin{figure}[h]
	\includegraphics[width=0.4\textwidth]{3.PNG}
\end{figure}
\fbox{
	\parbox{17cm}{На рисунке изображены графики функций $f(x)=a\sqrt{x-b}+c$ и $g(x)=kx-d$,
	 которые пересекаются в точках A и B. Найдите ординату точки B.
	 
	 $f(x)=2\sqrt{x+0}-3$

	 $g(x)=0,5x-1,5$

	 $A(1;-1)$

	 $B(9;3)$
	 }}
%%%%%%%%%%%%%%%%%%%%%

\vspace{\baselineskip}

\lstinputlisting{codes/ex_509123.js}
Примеры генерируемых задач:

\begin{figure}[h]
	\includegraphics[width=0.4\textwidth]{4.PNG}
\end{figure}
\fbox{
	\parbox{17cm}{
		На рисунке изображён график функции $f(x)=a \ctg x+b$. Найдите $b$.

		$f(x)=2\ctg x-2$
	}}

\vspace{\baselineskip}
\newpage
\begin{figure}[h]
	\includegraphics[width=0.4\textwidth]{5.PNG}
\end{figure}
\fbox{
	\parbox{17cm}{
		На рисунке изображён график функции $f(x)=a \cos x+b$. Найдите $a$.

		$f(x)=-4\cos x+2$
	}}

\vspace{\baselineskip}
\begin{figure}[h]
	\includegraphics[width=0.4\textwidth]{6.PNG}
\end{figure}
\fbox{
	\parbox{17cm}{
		На рисунке изображён график функции $f(x)=a \sin x+b$. Найдите $b$.
		$f(x)=6 \sin x-3$
	}}
	%%sin прямым шрифтом