\section{Глава вторая}
\subsection{Функции используемые в проекте}
В ходе работы над проектом "Час ЕГЭ" была создана нестандартная библиотека для упрощения многих задач. Далее представлены наиболее используемые из них.

\textbf{Работа с массивами}

\textbf{Многочлены}
Элементы одномерного массива Array - коэффициенты, стоящие в порядке возрастания степеней.

\texttt{\textcolor{Orange}{Array}.\textcolor{Blue}{prototype}.\textcolor{Purple}{mn\_proizv}=\textcolor{Red}{function}()}

Находит производную от многочлена.
\texttt{\textcolor{Orange}{Array}.\textcolor{Blue}{prototype}.\textcolor{Purple}{mn\_vychisl}=\textcolor{Red}{function}()}

Находит корни многочлена.

\texttt{
\textcolor{Orange}{Array}.\textcolor{Blue}{prototype}.\textcolor{Purple}{mn\_txt}=\textcolor{Red}{function}()}

TeX-представление многочлена.%%%???

\texttt{
	\textcolor{Orange}{Array}.\textcolor{Blue}{prototype}.\textcolor{Purple}{mn\_pervoobr}=\textcolor{Red}{function}()
}

Находит первообразную от многочлена.%%TODO в чёр разница то??

\texttt{
	\textcolor{Orange}{Array}.\textcolor{Blue}{prototype}.\textcolor{Purple}{mn\_txtmas}=\textcolor{Red}{function}()
}

TeX-представление многочлена.

\texttt{
	\textcolor{Orange}{Array}.\textcolor{Blue}{prototype}.\textcolor{Purple}{mt\_pryam}=\textcolor{Red}{function}()
}

Возвращает коэффициенты a и b прямой y=ax+b, проходящей через две первые точки.

\textbf{Вспомогательные функции для массивов}

\texttt{
	\textcolor{Orange}{Array}.\textcolor{Blue}{prototype}.\textcolor{Purple}{shuffle}=\textcolor{Red}{function}(b)
}

Перемешивает массив случайным образом. Если b, то ещё и рекурсивно на один уровень.

\hypertarget{iz}{\texttt{
	\textcolor{Orange}{Array}.\textcolor{Blue}{prototype}.\textcolor{Purple}{iz}=\textcolor{Red}{function}(p1)
}}

Если p1 опущено, возвращает случайный элемент массива, иначе последовательность p1 неповторяющихся элементов массива.

\textbf{Матрицы}

\texttt{
	\textcolor{Red}{function} \textcolor{Purple}{multiplyMatrix}(A,B)
}

Умножает матрицу A на B, возвращает результат в матрице C.

\texttt{
	\textcolor{Red}{function} \textcolor{Purple}{Determinant}(A)
}

Возвращает определитель матрицы A.

\texttt{
	\textcolor{Red}{function} \textcolor{Purple}{MatrixCofactor}(i,j,A)
}

Возвращает алгебраическое дополнение матрицы A.

\texttt{
	\textcolor{Red}{function} \textcolor{Purple}{AdjugateMatrix}(A)
}

Возвращает союзную(присоединенную) матрицу.

\texttt{
	\textcolor{Red}{function} \textcolor{Purple}{rang\_mat}(А)
}

Возвращает ранг матрицы A.

\texttt{
	\textcolor{Red}{function} \textcolor{Purple}{InverseMatrix}(B)
}

Возвращает обратную  к B матрицу.

\texttt{
	\textcolor{Red}{function} \textcolor{Purple}{generateMatrix}(stroki,stolbcy,min,max,p1)
}

Генерирует матрицу из случайных чисел.

\textbf{Работа с числами}

\texttt{
	\textcolor{Orange}{Number}.\textcolor{Blue}{prototype}.\textcolor{Purple}{pow}=\textcolor{Red}{function}(n)
}

Возвращает число в степени n.

\texttt{
	\textcolor{Orange}{Number}.\textcolor{Blue}{prototype}.\textcolor{Purple}{sqrt}=\textcolor{Red}{function}(n)
}

Возвращает квадратный корень из числа.

\texttt{
	\textcolor{Orange}{Number}.\textcolor{Blue}{prototype}.\textcolor{Purple}{sqr}=\textcolor{Red}{function}()
}

Возвращает квадрат числа.

\texttt{
	\textcolor{Orange}{Number}.\textcolor{Blue}{prototype}.\textcolor{Purple}{abs}=\textcolor{Red}{function}()
}

Возвращает модуль числа.

\texttt{
	\textcolor{Orange}{Number}.\textcolor{Blue}{prototype}.\textcolor{Purple}{floor}=\textcolor{Red}{function}()
}

Возвращает число, округленное до целого в меньшую сторону.

\texttt{
	\textcolor{Orange}{Number}.\textcolor{Blue}{prototype}.\textcolor{Purple}{ceil}=\textcolor{Red}{function}()
}

Возвращает число, округленное до целого в большую сторону.

\texttt{
	\textcolor{Orange}{Number}.\textcolor{Blue}{prototype}.\textcolor{Purple}{pm}=\textcolor{Red}{function}()
}

Случайным образом возвращает число или ему противоположное.

\texttt{
	\textcolor{Orange}{Number}.\textcolor{Blue}{prototype}.\textcolor{Purple}{ts}=\textcolor{Red}{function}()
}

Приводит число к стандартному виду (с десятичной запятой и не более чем 10 знаками после неё).

\texttt{
	\textcolor{Orange}{Number}.\textcolor{Blue}{prototype}.\textcolor{Purple}{texfracpi}=\textcolor{Red}{function}(p1)
}

Возвращает TeX-представление дроби, у которой в числителе данное число, умноженное на $\pi$, а в знаменателе p1.
Случай p1=1 учитывается.

\texttt{
	\textcolor{Orange}{Number}.\textcolor{Blue}{prototype}.\textcolor{Purple}{texsqrt}=\textcolor{Red}{function}(p1,p2)
}

TeX-представление корня из данного числа.
Если данное число - полный квадрат, то корень из числа.
Если p1, то из-под корня будут вынесены возможные множители.
Если p1, p2 и из-под корня выносится единица, то она будет опущена.

\texttt{
	\textcolor{Orange}{Number}.\textcolor{Blue}{prototype}.\textcolor{Purple}{isZ}=\textcolor{Red}{function}()
}

Возвращает true, если число n целое.

\texttt{
	\textcolor{Orange}{Number}.\textcolor{Blue}{prototype}.\textcolor{Purple}{isPolnKvadr}=\textcolor{Red}{function}()
}

Возвращает true, если число является полным квадратом.

\textbf{Работа с текстом}

\hypertarget{toZagl}{\texttt{
	\textcolor{Orange}{Number}.\textcolor{Blue}{prototype}.\textcolor{Purple}{toZagl}=\textcolor{Red}{function}()
}}

Возвращает исходную строку с первой заглавной буквой.

\texttt{
	\textcolor{Orange}{Number}.\textcolor{Blue}{prototype}.\textcolor{Purple}{esli}=\textcolor{Red}{function}()
}

Возвращает данную строку, если p1, и пустую в противном случае.

\texttt{
	\textcolor{Orange}{Number}.\textcolor{Blue}{prototype}.\textcolor{Purple}{plusminus}=\textcolor{Red}{function}()
}

Возвращает упрощенное выражение, вставляя между числами необходимые знаки и убирая нулевые.

\textbf{Работа с canvas}

\texttt{
	\textcolor{Orange}{CanvasRenderingContext2D}.\textcolor{Blue}{prototype}.\textcolor{Purple}{drawLine}=\textcolor{Red}{function}(x1,y1,x2,y2)
}

Рисует линию из точки (x1,y1) в (x2,y2).

\texttt{
	\textcolor{Orange}{CanvasRenderingContext2D}.\textcolor{Blue}{prototype}.\textcolor{Purple}{fillKrug}=\textcolor{Red}{function}(x,y,r)
}

Рисует круг с центром в (x,y) и радиусом r.

\texttt{
	\textcolor{Orange}{CanvasRenderingContext2D}.\textcolor{Blue}{prototype}.\textcolor{Purple}{drawArrow}=\textcolor{Red}{function}(x1, y1, x2, y2, arrowType)
}%%TODO Узнать зачем arrowType

Рисует стрелку из точки (x1,y1) в (x2,y2).

\hypertarget{drawCoordPlane}{\texttt{
	\textcolor{Orange}{CanvasRenderingContext2D}.\textcolor{Blue}{prototype}.\textcolor{Purple}{drawCoordPlane }=\textcolor{Red}{function}(w, h, kl, text, mash)
}}

Рисует координатную плоскость. w и h  \-- её размеры, kl \-- объект с полями hor и ver(высота и ширина клетки), text \-- объект с полями x1 и y1(единичные отрезки типа string), sh1 и sh2 (шрифты для x1, y1, по умолчанию 12px) и mash - масштаб изображения (по умолчанию равно 1).

\texttt{
	\textcolor{Orange}{CanvasRenderingContext2D}.\textcolor{Blue}{prototype}.\textcolor{Purple}{setkaVer2}=
	\newline
	\textcolor{Red}{function}(h, w, hor, ver, mash)
}

Рисует прямоугольную сетку. w и h  \-- её размеры, hor и ver \-- высота и ширина клетки, mash - масштаб (по умолчанию равно 1).

\hypertarget{graph9AmarkCircles}{\texttt{
	\textcolor{Red}{function} \textcolor{Purple}{graph9AmarkCircles}(ct, XY, maxQuantity, radius)
}}

Рисует maxQuantity точек в координатах из двумерно массива XY радиусом radius.
%%Переписать!

\hypertarget{graph9AdrawFunction}{\texttt{
	\textcolor{Red}{function} \textcolor{Purple}{graph9AdrawFunction}(ct, f, o)
}}

Рисует график f(x). Границы отображения задаются объектом o с полями maxX, maxY, minX, minY.

\textbf{Вспомогательные функции}
\hypertarget{sluchch}{\texttt{
	\textcolor{Red}{function} \textcolor{Purple}{sluchch}(n,k,s)
}}

Возвращает случайное число от n до k с шагом s по умолчанию 1).

\hypertarget{slKrome}{\texttt{
	\textcolor{Red}{function} \textcolor{Purple}{slKrome}(a,p1,p2,p3)
}}

Возвращает случайное число, кроме a. Если a \-- массив, то не содержащееся в нём; Если число или строка, то не равное ему; Если функция, принимающая параметр - то не удовлетворяющее ей.

\texttt{
	\textcolor{Red}{function} \textcolor{Purple}{sluchDel}(a)
}

Случайный делитель числа a.

\texttt{
	\textcolor{Red}{function} \textcolor{Purple}{sluchiz}(a,n)
}

Возвращает массив из n случайных не повторяющихся элементов массива a.

\texttt{
	\textcolor{Red}{function} \textcolor{Purple}{slLetter}(b)
}

Возвращает случайную букву английского алфавита.

\hypertarget{intPoints}{\texttt{
	\textcolor{Red}{function} \textcolor{Purple}{intPoints}(f,o)
}}

Возвращает двумерный массив из всех целых точек графика f(x). Границы нахождения точек задаются объектом o с полями maxX, maxY, minX, minY.

\subsection{Вклад автора в расширение каталога}