\documentclass[a4paper, 14pt]{extarticle}
\usepackage{fontspec}
\setmainfont{CMU Serif}[Ligatures=TeX]
\setmonofont{CMU Typewriter Text}
\usepackage{polyglossia}
\setdefaultlanguage{russian}
\usepackage[left=1cm,right=1cm,
  top=2cm,bottom=2cm]{geometry}
%%% Дополнительная работа с математикой
\usepackage{amsfonts,amssymb,amsthm,mathtools} % AMS
\usepackage{amsmath}
\usepackage{icomma} % "Умная" запятая: $0,2$ --- число, $0, 2$ --- перечисление

\usepackage{mathrsfs} % Красивый матшрифт

%% Перенос знаков в формулах (по Львовскому)
\newcommand*{\hm}[1]{#1\nobreak\discretionary{}
  {\hbox{$\mathsurround=0pt #1$}}{}}

%%% Работа с картинками
\usepackage{graphicx}  % Для вставки рисунков
\graphicspath{image}  % папки с картинками
\setlength\fboxsep{3pt} % Отступ рамки \fbox{} от рисунка
\setlength\fboxrule{1pt} % Толщина линий рамки \fbox{}
\usepackage{wrapfig} % Обтекание рисунков и таблиц текстом

%%% Работа с таблицами
\usepackage{array,tabularx,tabulary,booktabs} % Дополнительная работа с таблицами
\usepackage{longtable}  % Длинные таблицы
\usepackage{multirow} % Слияние строк в таблице
\usepackage{alltt}

\usepackage[dvipsnames]{xcolor}
\usepackage{verbatim}
\usepackage{hyperref}
\usepackage{listings}

\lstdefinelanguage{JavaScript}{
  keywords={let, typeof, new, true, false, catch, function, return, null, catch, switch, var, if, in, while, do, else, case, break},
  keywordstyle=\color{blue}\bfseries,
  ndkeywords={class, export, boolean, throw, implements, import, this},
  ndkeywordstyle=\color{darkgray}\bfseries,
  identifierstyle=\color{black},
  sensitive=false,
  comment=[l]{//},
  morecomment=[s]{/*}{*/},
  commentstyle=\color{purple}\ttfamily,
  stringstyle=\color{red}\ttfamily,
  morestring=[b]',
  morestring=[b]"
}

\lstset{
  language=JavaScript,
  extendedchars=true,
  basicstyle=\footnotesize\ttfamily,
  showstringspaces=false,
  breakatwhitespace=true,
  showspaces=false,
  numbers=left,
  numberstyle=\footnotesize,
  numbersep=9pt,
  tabsize=2,
  keepspaces=true,
  breaklines=true,
  showtabs=false,
  captionpos=b
  escapechar =| ,
  frame=single ,
  commentstyle=\itshape ,
  stringstyle =\bfseries
}
\usepackage{graphicx}
\graphicspath{ {Grafs/} }


\begin{document}
	\title{Важные функции}
	\section{Работа с массивами}
		\subsection{Многочлены}
		Элементы одномерного массива Array - коэффициенты, стоящие в порядке возрастания степеней.
			\begin{alltt}
				\textcolor{Orange}{Array}.\textcolor{Blue}{prototype}.\textcolor{Purple}{mn_proizv}=\textcolor{Red}{function}(
			\end{alltt}
		Находит производную от многочлена. 
			\begin{alltt}
				\textcolor{Orange}{Array}.\textcolor{Blue}{prototype}.\textcolor{Purple}{mn_vychisl}=\textcolor{Red}{function}()
			\end{alltt}
		Находит корни многочлена.
			\begin{alltt}
				\textcolor{Orange}{Array}.\textcolor{Blue}{prototype}.\textcolor{Purple}{mn_txt}=\textcolor{Red}{function}()
			\end{alltt}
		TeX-представление многочлена.%%%???
			\begin{alltt}
				\textcolor{Orange}{Array}.\textcolor{Blue}{prototype}.\textcolor{Purple}{mn_pervoobr}=\textcolor{Red}{function}()
			\end{alltt}
		Находит первообразную от многочлена.%%TODO в чёр разница то??
			\begin{alltt}
				\textcolor{Orange}{Array}.\textcolor{Blue}{prototype}.\textcolor{Purple}{mn_txtmas}=\textcolor{Red}{function}()
			\end{alltt}
		TeX-представление многочлена.
			\begin{alltt}
				\textcolor{Orange}{Array}.\textcolor{Blue}{prototype}.\textcolor{Purple}{mt_pryam}=\textcolor{Red}{function}()
			\end{alltt}
		Возвращает коэффициенты a и b прямой y=ax+b, проходящей через две первые точки.
		\subsection{Вспомогательные функции для массивов}
			\begin{alltt}
				\textcolor{Orange}{Array}.\textcolor{Blue}{prototype}.\textcolor{Purple}{shuffle}=\textcolor{Red}{function}(b)
			\end{alltt}	
		Перемешивает массив случайным образом. Если b, то ещё и рекурсивно на один уровень.
			\begin{alltt}
				\textcolor{Orange}{Array}.\textcolor{Blue}{prototype}.\textcolor{Purple}{iz}=\textcolor{Red}{function}(p1)
			\end{alltt}
		Если p1 опущено, возвращает случайный элемент массива, иначе последовательность p1 неповторяющихся элементов массива.
		\subsection{Матрицы}
			\begin{alltt} 	
				\textcolor{Red}{function} \textcolor{Purple}{multiplyMatrix}(A,B)
			\end{alltt}
		Умножает матрицу A на B, возвращает результат в матрице C.
			\begin{alltt} 	
				\textcolor{Red}{function} \textcolor{Purple}{Determinant}(A)
			\end{alltt}
		Возвращает определитель матрицы A.
			\begin{alltt} 
				\textcolor{Red}{function} \textcolor{Purple}{MatrixCofactor}(i,j,A)
			\end{alltt}
		Возвращает алгебраическое дополнение матрицы A.
			\begin{alltt} 
				\textcolor{Red}{function} \textcolor{Purple}{AdjugateMatrix}(A)
			\end{alltt}
		Возвращает союзную(присоединенную) матрицу.
			\begin{alltt} 	
				\textcolor{Red}{function} \textcolor{Purple}{rang_mat}(А)
			\end{alltt}
		Возвращает ранг матрицы A.
			\begin{alltt} 	
				\textcolor{Red}{function} \textcolor{Purple}{InverseMatrix}(B)
			\end{alltt}
		Возвращает обратную  к B матрицу.
			\begin{alltt} 	
				\textcolor{Red}{function} \textcolor{Purple}{generateMatrix}(stroki,stolbcy,min,max,p1)
			\end{alltt}
		Генерирует матрицу из случайных чисел.
	\section{Работа с числами}
		\begin{alltt}
			\textcolor{Orange}{Number}.\textcolor{Blue}{prototype}.\textcolor{Purple}{pow}=\textcolor{Red}{function}(n)
		\end{alltt}
	Возвращает число в степени n.
		\begin{alltt}
			\textcolor{Orange}{Number}.\textcolor{Blue}{prototype}.\textcolor{Purple}{sqrt}=\textcolor{Red}{function}(n)
		\end{alltt}
	Возвращает квадратный корень из числа.
		\begin{alltt}
			\textcolor{Orange}{Number}.\textcolor{Blue}{prototype}.\textcolor{Purple}{sqr}=\textcolor{Red}{function}()
		\end{alltt}
	Возвращает квадрат числа.
		\begin{alltt}
			\textcolor{Orange}{Number}.\textcolor{Blue}{prototype}.\textcolor{Purple}{abs}=\textcolor{Red}{function}()
		\end{alltt}
	Возвращает модуль числа.
		\begin{alltt}
			\textcolor{Orange}{Number}.\textcolor{Blue}{prototype}.\textcolor{Purple}{floor}=\textcolor{Red}{function}()
		\end{alltt}
	Возвращает число, округленное до целого в меньшую сторону.
		\begin{alltt}
			\textcolor{Orange}{Number}.\textcolor{Blue}{prototype}.\textcolor{Purple}{ceil}=\textcolor{Red}{function}()
		\end{alltt}
	Возвращает число, округленное до целого в большую сторону.
		\begin{alltt}
			\textcolor{Orange}{Number}.\textcolor{Blue}{prototype}.\textcolor{Purple}{pm}=\textcolor{Red}{function}()
		\end{alltt}
	Случайным образом возвращает число или ему противоположное.
		\begin{alltt}
			\textcolor{Orange}{Number}.\textcolor{Blue}{prototype}.\textcolor{Purple}{ts}=\textcolor{Red}{function}()
		\end{alltt}
	Приводит число к стандартному виду (с десятичной запятой и не более чем 10 знаками после неё).
		\begin{alltt}
			\textcolor{Orange}{Number}.\textcolor{Blue}{prototype}.\textcolor{Purple}{texfracpi}=\textcolor{Red}{function}(p1)
		\end{alltt}
	Возвращает TeX-представление дроби, у которой в числителе данное число, умноженное на $\pi$, а в знаменателе p1.
	Случай p1=1 учитывается.
		\begin{alltt}
			\textcolor{Orange}{Number}.\textcolor{Blue}{prototype}.\textcolor{Purple}{texsqrt}=\textcolor{Red}{function}(p1,p2)
		\end{alltt}	
	TeX-представление корня из данного числа.
	Если данное число - полный квадрат, то корень из числа.
	Если p1, то из-под корня будут вынесены возможные множители.
	Если p1, p2 и из-под корня выносится единица, то она будет опущена.
		\begin{alltt}
			\textcolor{Orange}{Number}.\textcolor{Blue}{prototype}.\textcolor{Purple}{isZ}=\textcolor{Red}{function}()
		\end{alltt}	
	Возвращает true, если число n целое.
		\begin{alltt}
			\textcolor{Orange}{Number}.\textcolor{Blue}{prototype}.\textcolor{Purple}{isPolnKvadr}=\textcolor{Red}{function}()
		\end{alltt}	
	Возвращает true, если число является полным квадратом.
	\section{Работа с текстом}
		\begin{alltt}
			\textcolor{Orange}{Number}.\textcolor{Blue}{prototype}.\textcolor{Purple}{toZagl}=\textcolor{Red}{function}()
		\end{alltt}	
	Возвращает исходную строку с первой заглавной буквой.
		\begin{alltt}
			\textcolor{Orange}{Number}.\textcolor{Blue}{prototype}.\textcolor{Purple}{esli}=\textcolor{Red}{function}()
		\end{alltt}	
	Возвращает данную строку, если p1, и пустую в противном случае.
		\begin{alltt}
			\textcolor{Orange}{Number}.\textcolor{Blue}{prototype}.\textcolor{Purple}{plusminus}=\textcolor{Red}{function}()
		\end{alltt}	
	Возвращает упрощенное выражение, вставляя между числами необходимые знаки и убирая нулевые. 
	\section{Работа с canvas}
		\begin{alltt}
			\textcolor{Orange}{CanvasRenderingContext2D}.\textcolor{Blue}{prototype}.\textcolor{Purple}{drawLine}=\textcolor{Red}{function}(x1,y1,x2,y2)
		\end{alltt}
	Рисует линию из точки (x1,y1) в (x2,y2).
		\begin{alltt}
			\textcolor{Orange}{CanvasRenderingContext2D}.\textcolor{Blue}{prototype}.\textcolor{Purple}{fillKrug}=\textcolor{Red}{function}(x,y,r)
		\end{alltt}
	Рисует круг с центром в (x,y) и радиусом r. 
		\begin{alltt}
			\textcolor{Orange}{CanvasRenderingContext2D}.\textcolor{Blue}{prototype}.\textcolor{Purple}{drawArrow}=\textcolor{Red}{function}(x1, y1, x2, y2, arrowType)
		\end{alltt}%%TODO Узнать зачем arrowType
	Рисует стрелку из точки (x1,y1) в (x2,y2).
		\begin{alltt}
			\textcolor{Orange}{CanvasRenderingContext2D}.\textcolor{Blue}{prototype}.\textcolor{Purple}{drawCoordPlane }=\textcolor{Red}{function}(w, h, kl, text, mash)
		\end{alltt}
	Рисует координатную плоскость. w и h  \-- её размеры, kl \-- объект с полями hor и ver(высота и ширина клетки), text \-- объект с полями x1 и y1(единичные отрезки типа string), sh1 и sh2 (шрифты для x1, y1, по умолчанию 12px) и mash - масштаб изображения (по умолчанию равно 1).
		\begin{alltt}
			\textcolor{Orange}{CanvasRenderingContext2D}.\textcolor{Blue}{prototype}.\textcolor{Purple}{setkaVer2  }=\textcolor{Red}{function}(h, w, hor, ver, mash)
		\end{alltt}
	Рисует прямоугольную сетку. w и h  \-- её размеры, hor и ver \-- высота и ширина клетки, mash - масштаб (по умолчанию равно 1).
		\begin{alltt} 
			\textcolor{Red}{function} \textcolor{Purple}{graph9AmarkCircles}(ct, XY, maxQuantity, radius)
		\end{alltt}	
	Рисует maxQuantity точек в координатах из двумерно массива XY радиусом radius.
	%%Переписать!
		\begin{alltt}
			\textcolor{Red}{function} \textcolor{Purple}{graph9AdrawFunction}(ct, f, o)
		\end{alltt}	
	Рисует график f(x). Границы отображения задаются объектом o с полями maxX, maxY, minX, minY.
	\section{Вспомогательные функции}
		\begin{alltt} 	
			\textcolor{Red}{function} \textcolor{Purple}{sluchch}(n,k,s)
		\end{alltt}
	Возвращает случайное число от n до k с шагом s(по умолчанию 1).
		\begin{alltt} 	
			\textcolor{Red}{function} \textcolor{Purple}{slKrome}(a,p1,p2,p3)
		\end{alltt}
	Возвращает случайное число, кроме a. Если a \-- массив, то не содержащееся в нём; Если число или строка, то не равное ему; Если функция, принимающая параметр - то не удовлетворяющее ей.
		\begin{alltt} 	
			\textcolor{Red}{function} \textcolor{Purple}{sluchDel}(a)
		\end{alltt}
	Случайный делитель числа a.
		\begin{alltt} 	
			\textcolor{Red}{function} \textcolor{Purple}{sluchiz}(a,n)
		\end{alltt}
	Возвращает массив из n случайных не повторяющихся элементов массива a. 
		\begin{alltt} 	
			\textcolor{Red}{function} \textcolor{Purple}{slLetter}(b)
		\end{alltt}
	Возвращает случайную букву английского алфавита.
		\begin{alltt} 	
			\textcolor{Red}{function} \textcolor{Purple}{intPoints}(f,o)
		\end{alltt}
  	Возвращает двумерный массив из всех целых точек графика f(x).  Границы нахождения точек задаются объектом o с полями maxX, maxY, minX, minY.
  	
\end{document}
