\documentclass[a4paper, 14pt]{extarticle}
\usepackage{fontspec}
\setmainfont{CMU Serif}[Ligatures=TeX]
\setmonofont{CMU Typewriter Text}
\usepackage{polyglossia}
\setdefaultlanguage{russian}
\usepackage[left=1cm,right=1cm,
  top=2cm,bottom=2cm]{geometry}
%%% Дополнительная работа с математикой
\usepackage{amsfonts,amssymb,amsthm,mathtools} % AMS
\usepackage{amsmath}
\usepackage{icomma} % "Умная" запятая: $0,2$ --- число, $0, 2$ --- перечисление

\usepackage{mathrsfs} % Красивый матшрифт

%% Перенос знаков в формулах (по Львовскому)
\newcommand*{\hm}[1]{#1\nobreak\discretionary{}
  {\hbox{$\mathsurround=0pt #1$}}{}}

%%% Работа с картинками
\usepackage{graphicx}  % Для вставки рисунков
\graphicspath{image}  % папки с картинками
\setlength\fboxsep{3pt} % Отступ рамки \fbox{} от рисунка
\setlength\fboxrule{1pt} % Толщина линий рамки \fbox{}
\usepackage{wrapfig} % Обтекание рисунков и таблиц текстом

%%% Работа с таблицами
\usepackage{array,tabularx,tabulary,booktabs} % Дополнительная работа с таблицами
\usepackage{longtable}  % Длинные таблицы
\usepackage{multirow} % Слияние строк в таблице
\usepackage{alltt}

\usepackage[dvipsnames]{xcolor}
\usepackage{verbatim}
\usepackage{hyperref}
\usepackage{listings}

\lstdefinelanguage{JavaScript}{
  keywords={let, typeof, new, true, false, catch, function, return, null, catch, switch, var, if, in, while, do, else, case, break},
  keywordstyle=\color{blue}\bfseries,
  ndkeywords={class, export, boolean, throw, implements, import, this},
  ndkeywordstyle=\color{darkgray}\bfseries,
  identifierstyle=\color{black},
  sensitive=false,
  comment=[l]{//},
  morecomment=[s]{/*}{*/},
  commentstyle=\color{purple}\ttfamily,
  stringstyle=\color{red}\ttfamily,
  morestring=[b]',
  morestring=[b]"
}

\lstset{
  language=JavaScript,
  extendedchars=true,
  basicstyle=\footnotesize\ttfamily,
  showstringspaces=false,
  breakatwhitespace=true,
  showspaces=false,
  numbers=left,
  numberstyle=\footnotesize,
  numbersep=9pt,
  tabsize=2,
  keepspaces=true,
  breaklines=true,
  showtabs=false,
  captionpos=b
  escapechar =| ,
  frame=single ,
  commentstyle=\itshape ,
  stringstyle =\bfseries
}
\usepackage{graphicx}
\graphicspath{ {Grafs/} }


\begin{document}
\section{Введение}%%да простит меня автор за плагиат
«Час ЕГЭ» — компьютерный образовательный проект, разрабатываемый при математическом факультете ВГУ в рамках «OpenSource кластера» и предназначенный для помощи учащимся старших классов подготовиться к тестовой части единого государственного экзамена.

Задания в «Час ЕГЭ» генерируются случайным образом по специализированным алгоритмам, называемых шаблонами. Каждый из которых охватывает множество вариантов соответствующей ему задачи. Для пользователей предназначены четыре оболочки (режима работы): «Случайное задание», «Тесты на печать», «Полный тест» и «Мини-интеграция».

«Час ЕГЭ» является полностью открытым (код находится под лицензией GNU GPL 3.0) и бесплатным. В настоящее время в проекте полностью реализованы тесты по математике с кратким ответом (бывшая «часть В»). Планируется с течением времени включить в проект тесты по другим предметам школьной программы.

«Мини-интеграция» — это форма сотрудничества с образовательными интернет-ресурсами, при которой учебно-методический материал на странице ресурса дополняется виджетами тренажера с заданиями, соответствующими теме статьи, для возможности практического применения полученных знаний. В настоящее время достигнуто сотрудничество с двумя образовательными ресурсами: ege-ok.ru и matematikalegko.ru .

Целью работы является расширение каталога заданий на основе "Открытого банка заданий ЕГЭ" и дополнение его альтернативными задачами	на основе имеющихся.

Разработанный авторами тренажёр способен генерировать не ограниченное количество заданий каждого типа, что позволяет школьнику упражняться в их решении, пока он не посчитает нужным закончить, а не пока все задания, предоставляемые иными источниками, закончатся.

В тренажёре является гибким к изменениям в самом экзамене, шаблоны легко добавить или изменить для соответствия реальному ЕГЭ. Так же при появлении новых заданий школьник больше не должен ожидать появления его в сборниках или на сайтах в достаточном количестве для прорешивания и разбора.

\end{document}
