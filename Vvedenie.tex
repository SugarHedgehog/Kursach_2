\documentclass[a4paper,12pt]{article}
%%% Работа с русским языком
\usepackage{cmap}					% поиск в PDF
\usepackage{mathtext} 				% русские буквы в фомулах
\usepackage[T2A]{fontenc}			% кодировка
\usepackage[utf8]{inputenc}			% кодировка исходного текста
\usepackage[english,russian]{babel}	% локализация и переносы
\begin{document}
	\section{Введение}%%да простит меня автор за плагиат
	«Час ЕГЭ» — компьютерный образовательный проект, разрабатываемый при математическом факультете ВГУ в рамках «OpenSource кластера» и предназначенный для помощи учащимся старших классов подготовиться к тестовой части единого государственного экзамена. 
	
	Задания в «Час ЕГЭ» генерируются случайным образом по специализированным алгоритмам, называемых шаблонами. Каждый из которых охватывает множество вариантов соответствующей ему задачи. Для пользователей предназначены четыре оболочки (режима работы): «Случайное задание», «Тесты на печать», «Полный тест» и «Мини-интеграция».
	
	«Час ЕГЭ» является полностью открытым (код находится под лицензией GNU GPL 3.0) и бесплатным. В настоящее время в проекте полностью реализованы тесты по математике с кратким ответом (бывшая «часть В»). Планируется с течением времени включить в проект тесты по другим предметам школьной программы.
	
	«Мини-интеграция» — это форма сотрудничества с образовательными интернет-ресурсами, при которой учебно-методический материал на странице ресурса дополняется виджетами тренажера с заданиями, соответствующими теме статьи, для возможности практического применения полученных знаний. В настоящее время достигнуто сотрудничество с двумя образовательными ресурсами: ege-ok.ru и matematikalegko.ru .
	
	Целью работы является расширение каталога заданий на основе "Открытого банка заданий ЕГЭ" и дополнение его альтернативными задачами	на основе имеющихся. 
	
	Разработанный авторами тренажёр способен генерировать не ограниченное количество заданий каждого типа, что позволяет школьнику упражняться в их решении, пока он не посчитает нужным закончить, а не пока все задания, предоставляемые иными источниками, закончатся. 
	
	В тренажёре является гибким к изменениям в самом экзамене, шаблоны легко добавить или изменить для соответствия реальному ЕГЭ. Так же при появлении новых заданий школьник больше не должен ожидать появления его в сборниках или на сайтах в достаточном количестве для прорешивания и разбора.
	
\end{document}
