
\section{Введение}%%TODO: определение ну такое себе
Единый государственный экзамен (ЕГЭ) --- централизованно проводимый в Российской Федерации экзамен в средних учебных заведениях — школах, лицеях и гимназиях,
форма проведения ГИА по образовательным программам среднего общего образования.
Служит одновременно выпускным экзаменом из школы и вступительным экзаменом в вузы.

ЕГЭ введён 2009 году и был множество раз изменён и дополнен. 
При этом печатные и электронные методические материалы для подготовки с этими изменениями приходили в негодность. А изменение и появление новых, достаточных для прорешивания, занимало и занимает огромное количество времени. 

Но одно неизменное качество остаётся в экзамене и в сегодняшний день - стандартные задания. 
Это и позволило появлению проекта "Час ЕГЭ".

«Час ЕГЭ» — компьютерный образовательный проект, разрабатываемый при математическом факультете ВГУ в рамках «OpenSource кластера» и предназначенный для помощи учащимся старших классов подготовиться к тестовой части единого государственного экзамена.

Задания в «Час ЕГЭ» генерируются случайным образом по специализированным алгоритмам, называемых шаблонами.
Каждый из которых охватывает множество вариантов соответствующей ему задачи. Для пользователей предназначены четыре оболочки (режима работы): «Случайное задание», «Тесты на печать», «Полный тест» и «Мини-интеграция».

«Час ЕГЭ» является полностью открытым (код находится под лицензией GNU GPL 3.0) и бесплатным.
В настоящее время в проекте полностью реализованы тесты по математике с кратким ответом (бывшая «часть В»).
Планируется с течением времени включить в проект тесты по другим предметам школьной программы.

«Мини-интеграция» — это форма сотрудничества с образовательными интернет-ресурсами, при которой учебно-методический материал на странице ресурса дополняется виджетами тренажера с заданиями, соответствующими теме статьи, для возможности практического применения полученных знаний.
В настоящее время достигнуто сотрудничество с двумя образовательными ресурсами: ege-ok.ru и matematikalegko.ru.

Цели работы:
\begin{enumerate}
    \item Расширение каталога заданий на основе "Открытого банка заданий ЕГЭ" и дополнение его альтернативными задачами на основе имеющихся.
    \item Выявление преимуществ генерации по средствам программных алгоритмов оценочных средств по математике.
    \item Систематизация и дополнение библиотеки функций проекта "Час ЕГЭ".
\end{enumerate}
